%% ----------------------------------------------------------------------------
% BIWI SA/MA thesis template
%
% Created 09/29/2006 by Andreas Ess
% Extended 13/02/2009 by Jan Lesniak - jlesniak@vision.ee.ethz.ch
%% ----------------------------------------------------------------------------

\chapter{Conclusion}

We were initially faced with the problem of reviewing photos from a mobile phone
photo collection, where the photos are used as wallpapers.
This is a two-fold problem concerning the selection of photos to use as
wallpapers, and centering or cropping the image appropriately such that
important or salient areas are shown in an aesthetically pleasing way on the
phone display.

When determining if an image could be used as a wallpaper, it is propagated
through a deep convolutional neural network trained to identify object
categories.
The neuron activations together with annotations are then used to train a SVM.
In practise it successfully detects and discards scenes such as those with text and furniture.
There however exists a bias towards distant scenes of nature where mountains, rivers, lakes and sky photos are favoured.
Further work could be done in matching personal tastes.

When trained on the Michael dataset and evaluated on the Wookie dataset, an
error rate of 3.7\% is observed where most images are classified correctly.
It is noted that for each annotator there is usually an improvement in precision
when training on just his/her own annotations.
This is to be expected as personal tastes would be better reflected in the
learning phase.
For the case of low correlation between annotations however, it would be
interesting to combine annotations selectively to yield a more aesthetically
pleasing final set of images.

Automatic cropping was selected for retargeting images to a specific aspect
ratio.
For cropping any given image, three cues are considered.
These are: saliency composition, boundary simplicity, and content preservation.
A SVM is trained with features based on these cues using a dataset sourced using
the Reddit web service.
The resulting model and algorithm works quite well, yielding a median maximum
overlap of 0.782 for top 1 crops.
This is an improvement over the previous state of the art \cite{fang2014automatic}.

Much more could be done to improve the suggested algorithms.

For instance, the datasets used in the selection stage could be improved with a
greater variety of photos as well as a larger number of both photos and
annotations.
Annotators could be asked to annotate based on several keywords or themes,
allowing for a classifier which attempts to adhere to user tastes.
More features could be added during the learning stage.
For instance, the colour distribution or blurriness of the image may change how
suitable an annotator finds the image.

The cropping algorithm shows issues especially in the case where the saliency
map or gradient map does not work as expected.
Future work on enhancing algorithms for generating the two maps should improve
the cropping algorithm.
Further on, image segmentation using SLIC superpixels for example could allow
for more advanced ways in respecting boundary simplicity as well as generating
better crop boundaries in general.

